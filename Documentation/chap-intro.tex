\chapter{Introduction}
\pagenumbering{arabic}%

The purpose of this work is to define a graphics layer that can be
used by \commonlisp{} toolkits for creating graphic user interfaces.
It can be thought of as an intermediate backend, sitting between the
GUI toolkit itself and some concrete backend such as X11, OpenGL,
Wayland, etc.

The basis of this work is the recognition that most of the graphics
operations of a graphics user interface are not performance critical.
We can therefore afford to simplify the graphics model for such
operations, while still allowing for client code to access
backend-specific rendering primitives for highly performance-critical
applications.

The purpose of this work is thus \emph{not} to replace all
backend-specific operations.  Instead, we allow for sophisticated
applications to obtain objects that can then be used in
backend-specific operations, thereby allowing such applications to
implement performance-critical operations such as real-time animation,
video rendering, etc.

However, a large number of graphics operations that a library for
graphic user interfaces needs to implement do not require very high
performance, nor any sophisticated rendering primitives.  The
operations defined here will allow such a library to define
backend-independent code for drawing most typical gadgets such as
buttons, scroll bars, menus, icons, etc.  Most importantly, it will
also contain primitives that can be used to draw text.

