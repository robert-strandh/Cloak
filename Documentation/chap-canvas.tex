\chapter{Canvas}
\label{chap-canvas}

\Defclass {canvas}

This class is the protocol class of all objects that can be targets of
drawing operations.

Each backend will define subclasses of this class, and each backend
will have its proper methods of creating instances of its own canvas
types.

\Defmacro {with-canvas} {(canvas-var canvas-form) \body body}

This macro is used to group several operations into a single final
rendering operation.

The argument \textit{canvas-form} is a form that evaluated once.  It
must evaluate to a canvas.

The argument \textit{canvas-var} is a symbol.  It is not evaluated.
In the body of the macro, this variable is bound to the result of
evaluating \textit{canvas-form}.

The \textit{body} can contain arbitrary forms, but the visible effect
of drawing operations is suspended until the end of the macro call.
Client code will typically wrap the entire event-handling machinery
inside a call to this macro.
