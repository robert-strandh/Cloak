\chapter{Clipping region}
\label{chap-clipping-region}

A \emph{clipping region} is an object that is used in order to limit
rendering to a subset of all the pixels in the plane.  

A clipping region can be seen as a simplified \emph{mask}.  Just like
a mask, it defines \emph{opacity} values for each pixel in the plane.
However, as opposed to a general mask, a clipping region defines each
pixel as being either completely transparent or completely opaque.

Furthermore, as opposed to a general mask, a clipping region is
\emph{bounded}, which means that, except for a bounded number of
pixels, it defines all pixels to be completely transparent.  This
feature is used by the backends to restrict the area that has to be
drawn.

\Defprotoclass{clipping-region}

This class is the protocol class for all clipping regions.

\Defclass {rectangle}

This class is a subclass of the class \texttt{clipping-region}.  It
defines a rectangular clipping region that is orthogonal to the axes
of the plane and which has integer coordinates.  All backends support
this type of clipping region.
