\chapter{Mask}

\Defprotoclass {mask}

A mask is a supplier of \emph{opacity} values.  For each pixel in the
plane, a mask object is capable of supplying an opacity value.
Different subclasses of this class have different representation.

Backends are allowed to treat mask instances as being immutable, so
that it can create caches based on identity (in the sense of
\texttt{eq}) of mask instances.  Therefore, even though a mask object
might be technically mutable, once it has been used in some operation
defined here, it should no longer be modified.

\Defclass {matrix-mask}

A mask of this type contains explicit storage for each opacity value
inside a rectangular subset of the plane.  For pixels outside that
rectangular subset, this mask type supplies an opacity value of 0d0,
i.e., totally transparent.

\Defclass {computed-mask}

A mask of this type contains a function that, when applied to a
horizontal and a vertical position, returns an opacity value.

\Defclass {triangle-mask}

A mask of this type contains a set of triangles, defined by their
corner coordinates.

\Definitarg {:triangles}

This initarg is used to supply a list of triangles (see below) that
will define the mask.

\Defclass {triangle}

This class is not itself a mask type.  An instance of this class just
contains three points in the plane.  A set (represented as a list) of
such instances is used to create a triangle mask.  The order of the
coordinates for the triangle corners in a triangle is not important.

\Definitarg {:x1}
\Definitarg {:y1}
\Definitarg {:x2}
\Definitarg {:y2}
\Definitarg {:x3}
\Definitarg {:y3}

These initargs are accepted by all subclasses of the class
\texttt{triangle}.
